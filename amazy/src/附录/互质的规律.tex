\small {
\textbf{互质规律:}比较常见的定义
1.较大数是质数, 两个数互质

2.较小数是质数, 较大数不是它的倍数, 两个数互质

3. 1与其他数互质

4. 2与奇数互质

一些推论
1. 两个相邻的自然数一定互质

2.两个相邻的奇数一定互质

3. n与2n + 1或2n - 1一定互质

求差判断法
如果两个数相差不大,可先求出它们的差,再看差与其中较小数是否互质。如果互质,则原来两个数一定是互质数。如:194和201,先求出它们的差,201-194=7,因7和194互质,则194和201是互质数。相反也成立, 对较大数也成立

求商判断法
用大数除以小数,如果除得的余数与其中较小数互质,则原来两个数是互质数。如:317和52,317÷52=6……5,因余数5与52互质,则317和52是互质数。\\

}