% 这里假设前面有:
% \section{数学}
%   ...
% \subsection{数论基础}   % 这是 7.5

\subsubsection{欧拉函数} % 7.5.1

\paragraph{定义}
$\varphi(n)$ 为 $1 \sim n$ 中和 $n$ 互质的数的个数。

\paragraph{性质}
若 $p$ 是质数:
\begin{itemize}
    \item $\varphi(p) = p - 1$
    \item $\varphi(p^{k}) = (p - 1)p^{k-1}$
\end{itemize}

积性函数:若 $\gcd(m, n) = 1$,则 $\varphi(mn) = \varphi(m)\varphi(n)$。

\paragraph{质数分解公式}
设 $n = \prod_{i = 1}^{s}p_i^{a_{i}}$,则
\[
\varphi(n) = n \times \frac{p_{1} - 1}{p_1} \times \cdots \times \frac{p_s - 1}{p_s}
          = n \prod_{i=1}^{s} \left(1 - \frac{1}{p_i}\right).
\]

\subsubsection{莫比乌斯函数} % 7.5.2

\paragraph{定义}
\[
\mu(n) =
\begin{cases}
1, & n = 1, \\[6pt]
0, & n \text{ 含有平方因子}, \\[6pt]
(-1)^k, & k \text{ 是 } n \text{ 的不同质因子的个数}.
\end{cases}
\]

\subsubsection{数论定理} % 7.5.3

\paragraph{费马小定理}
若 $p$ 为质数且 $\gcd(a,p)=1$,则
\[
a^{p-1} \equiv 1 \pmod{p}.
\]

\paragraph{欧拉定理}
若 $\gcd(a,n)=1$,则
\[
a^{\varphi(n)} \equiv 1 \pmod{n}.
\]

\paragraph{扩展欧拉定理}
\[
a^{b} \equiv 
\begin{cases}
a^{b}, & b < \varphi(n), \\[6pt]
a^{\,b \bmod \varphi(n) + \varphi(n)}, & b \ge \varphi(n),
\end{cases}
\pmod{n}
\]
常用于降幂。

\paragraph{威尔逊定理}
\[
(p - 1)! \equiv -1 \pmod{p}
\]
这是 $p$ 为素数的充要条件。若 $p>4$ 且为合数,则
\[
(p - 1)! \equiv 0 \pmod{p}.
\]
